\documentclass[a4paper,11pt]{article}
\usepackage[top=3.5cm, bottom=2.5cm, left=3cm, right=3cm]{geometry}
\usepackage[T1]{fontenc}
\usepackage[utf8]{inputenc}
\usepackage[frenchb]{babel}
\usepackage{ifpdf}
\usepackage{asymptote}
\usepackage{graphicx}
\usepackage{fancybox}
\usepackage{parallel}
\usepackage{amsmath}
\usepackage{array}
\usepackage{enumitem}
\usepackage{wallpaper}
\usepackage{listings}
\usepackage{alltt}
\usepackage{courier}

\lstset{
	basicstyle = \normalsize\ttfamily,
	tabsize=4,
}
\begin{document}

\URCornerWallPaper{.25}{/Users/Noe/Documents/Cours/S1/Ressources/Logo.png}
\begin{titlepage}

\newcommand{\HRule}{\rule{\linewidth}{0.5mm}} % Defines a new command for the horizontal lines, change thickness here

\center % Center everything on the page
 
%----------------------------------------------------------------------------------------
%	HEADING SECTIONS
%----------------------------------------------------------------------------------------

\textsc{\LARGE École Nationale des Ingénieurs de Brest}\\[1.5cm]
\textsc{\Large Document de Conception}\\[0.5cm]
\textsc{\large MDD-PROJET}\\[0.5cm]

%----------------------------------------------------------------------------------------
%	TITLE SECTION
%----------------------------------------------------------------------------------------

\HRule \\[0.4cm]
{\huge \bfseries Spazz}\\[0.4cm] % Title of your document
\HRule \\[1.5cm]
 
\Large
Noé \textsc{Maillard} et Allan \textsc{Dano}\\[3cm]


{Date}\\[3cm] % Date, change the \today to a set date if you want to be precise

{\large Version 1.0}
 

\vfill

\end{titlepage}

\tableofcontents
\newpage
\section{Rappel du cahier des charges}

\subsection{Contraintes techniques}

\begin{itemize}[label = $\bullet$]
	\item Le logiciel crée est évalué par les professeurs sur un ordinateur de salle de TP, il faut donc que le jeu s’exécute et soit jouable sur ces machines
	\item Le cours porte sur le langage Python, il est donc évident que le jeu soit écrit en Python
	\item Le paradigme utilisé est celui de la programmation procédurale
	\item L'interface doit être en mode texte dans le terminal
\end{itemize}

\subsection{Fonctionalités}

\begin{enumerate}[label*= F\arabic*,font = \textbf]
	\item : Choisir un pseudo
	\item : Choisir la difficulté
	\item : Jouer un niveau
	\begin{enumerate}[label*=.\arabic*,font = \textbf]
		\item : Choisir le niveau
		\item : Afficher le Jeu
		\item : Changer de direction
		\item : Ramasser un jeton
		\item : Finir le niveau
		\begin{enumerate}[label*=.\arabic*,font = \textbf]
			\item : Afficher résultat
			\item : Afficher les meilleurs scores
			\item : quitter le jeu
		\end{enumerate}
	\end{enumerate}
\end{enumerate}

\subsection{Prototype P1}

Ce prototype porte sur la création et l'affichage du niveau.\\
Mise en œuvre de fonctionnalités : F1, F2, F3.2, F3.2

\section{Principe des solutions techniques}

\subsection{Langage}

Conformément aux contraintes énoncées dans le cahier des charges, le codage est réalisé avec le langage Python.Nous choisissons la version 2.7.5.

\subsection{Architecture du logiciel}

Nous mettons en oeuvre le principe de la barrière d'abstraction. Chaque module correspond à un type de donnée et fournit toutes les opérations permettant de le manipuler de manière abstraite.

\subsection{Interface utilisateur}

L'interface utilisateur se fera via un terminal de type linux.\\

\subsubsection{Boucle de simulation}

Le programme mettra en oeuvre une boucle de simulation qui gérera l'affichage et et les événements clavier.

\subsubsection{Images ASCII-Art}

Pour stocker les niveaux du jeu nous utilisons des images ascii stockées dans des fichers textes

\newpage

\section{Analyse}

\subsection{Analyse noms/verbes}

\begin{lstlisting}
Verbes :

nommer, choisir, jouer, afficher, deplacer, finir, quitter

Noms :

joueur, Spazz, pseudo, direction, niveau, score, taille, position
\end{lstlisting}

\subsection{Types de Donnée}

\lstinputlisting{/Users/Noe/Documents/Cours/S2/MDD/Document_de_conception/Analyse_descendante/Types_de_donnees.txt}

\newpage

\subsection{Dépendance entre modules}

\subsection{Analyse descendante}

\subsubsection{Arbre principal}
    	
\begin{alltt}
\input{/Users/Noe/Documents/Cours/S2/MDD/Document_de_conception/Analyse_descendante/Arbre_principal.txt}
\end{alltt}

\subsubsection{Arbre réglages}

\begin{alltt}
\input{/Users/Noe/Documents/Cours/S2/MDD/Document_de_conception/Analyse_descendante/Arbre_Settings.txt}
\end{alltt}

\subsubsection{Arbre affichage}

\begin{alltt}
\input{/Users/Noe/Documents/Cours/S2/MDD/Document_de_conception/Analyse_descendante/Arbre_affichage.txt}
\end{alltt}
\subsubsection{Arbre interaction}

\begin{alltt}
	\input{/Users/Noe/Documents/Cours/S2/MDD/Document_de_conception/Analyse_descendante/Arbre_interaction.txt}
\end{alltt}
\newpage
\section{Description des fonctions}

\subsection{Programme principal : Main.py}

\begin{alltt}
	\input{/Users/Noe/Documents/Cours/S2/MDD/Document_de_conception/Description_des_fonctions/Main.txt}
\end{alltt}
\newpage
\subsection{Module Game.py}

\begin{alltt}
\input{/Users/Noe/Documents/Cours/S2/MDD/Document_de_conception/Description_des_fonctions/Game.txt}
\end{alltt}
\newpage
\subsection{Module Settings.py}

\begin{alltt}
	\input{/Users/Noe/Documents/Cours/S2/MDD/Document_de_conception/Description_des_fonctions/Settings.txt}
\end{alltt}

\subsection{Module Level.py}
\begin{alltt}
	\input{/Users/Noe/Documents/Cours/S2/MDD/Document_de_conception/Description_des_fonctions/Level.txt}
\end{alltt}

\newpage
\subsection{Module Coin.py}

\begin{alltt}
\input{/Users/Noe/Documents/Cours/S2/MDD/Document_de_conception/Description_des_fonctions/Coin.txt}
\end{alltt}

\section{Calendrier et suivi de développement}
\texttt{
\begin{tabular}{|l|c|c|r|}
\hline
fonctions & codées & testées & commentaires \\
\hline
	Settings.\textbf{askSettings} & Oui & Oui & 07/04 \\
\hline
	Settings.\textbf{askName} & Oui & Oui & 07/04\\
\hline
	Settings.\textbf{askDifficulty} & Oui & Oui & 07/04 \\
\hline 
	Settings.\textbf{askLevelNumber} & Oui & Oui & 07/04\\
\hline
\end{tabular}
}
\end{document}  